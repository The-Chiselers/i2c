\section{Port Descriptions}

\subsection{I2C Interface}

The ports for \textbf{I2C} are shown below in 
Table 1. The width of several ports is controlled 
by the following input parameters:

\renewcommand*{\arraystretch}{1.4}

\begingroup
\small

\rowcolors{2}{gray!30}{gray!10} 
\arrayrulecolor{gray!80}

\begin{longtable}[H]{
  | p{0.20\textwidth}
  | p{0.20\textwidth}
  | p{0.12\textwidth}
  | p{0.43\textwidth} |
}

\hline
\rowcolor{gray}
\textcolor{white}{\textbf{Port Name}} & 
\textcolor{white}{\textbf{Width}} & 
\textcolor{white}{\textbf{Direction}} & 
\textcolor{white}{\textbf{Description}} \\ \hline
\endfirsthead

\hline
\rowcolor{gray}
\textcolor{white}{\textbf{Port Name}} & 
\textcolor{white}{\textbf{Width}} & 
\textcolor{white}{\textbf{Direction}} & 
\textcolor{white}{\textbf{Description}}\\ \hline
\endhead

\hline
\endfoot

sclIn & 
1 & 
Input &
Input Clock Line to Master and Slave Interface. Used to synchronize transactions, detect clock stretching and for Multi-Master scenarios. \\ \hline

sclOut &
1 &
Output &
Output Clock Line from Master and Slave Interface. Used to drive transactions from the Master side and drive clock stretching. \\ \hline

sdaIn &
1 &
Input &
Input Data Line to Master and Slave Interface. Used by Master to read data from Slave. Used by Slave to read data from Master. \\ \hline

sdaOut &
1 &
Output &
Output Data Line from Master and Slave Interface. Used by Master to write data to Slave. Used by Slave to write data to Master. \\ \hline

\end{longtable}
\captionsetup{aboveskip=0pt}
\captionof{table}{I2C Ports Descriptions}\label{table:ports}
\endgroup

\subsection{Apb3 Interface}
The \textbf{Apb3 Interface} is a regular Apb3 Slave Interface. All signals supported are shown below in 
Table 2. See the \textit{AMBA Apb Protocol Specifications} for a complete description of the signals. The width of several ports is controlled 
by the following input parameters:

\begin{itemize}[noitemsep]
  \item \textit{dataWidth} is the width of PWDATA and PRDATA in bits
  \item \textit{addrWidth} is the width of PADDR in bits
\end{itemize}
 
\renewcommand*{\arraystretch}{1.4}

\begingroup
\small
\rowcolors{2}{gray!30}{gray!10} % Alternating colors start from the second row
\begin{longtable}[H]{
  | p{0.20\textwidth}
  | p{0.20\textwidth}
  | p{0.12\textwidth}
  | p{0.43\textwidth} |
  }

  \hline
  \rowcolor{gray}
  \textcolor{white}{\textbf{Port Name}} & 
  \textcolor{white}{\textbf{Width}} & 
  \textcolor{white}{\textbf{Direction}} & 
  \textcolor{white}{\textbf{Description}} \\ \hline
  \endfirsthead

  \hline
  \rowcolor{gray}
  \textcolor{white}{\textbf{Port Name}} & 
  \textcolor{white}{\textbf{Width}} & 
  \textcolor{white}{\textbf{Direction}} & 
  \textcolor{white}{\textbf{Description}}\\ \hline
  \endhead

  \hline
  \endfoot


  PCLK &       
  1 &       
  Input &       
  Positive edge clock \\ \hline

  PRESETN &       
  1 &       
  Input &       
  Active low reset \\ \hline

  PSEL &       
  1 & 
  Input &       
  Indicates slave is selected and a data transfer is required \\ \hline

  PENABLE &        
  1 & 
  Input &       
  Indicates second cycle of Apb transfer \\ \hline

  PWRITE &        
  1 & 
  Input &       
  Indicates write access when HIGH and read access when LOW\\ \hline

  PADDR &      
  \textit{addrWidth} & 
  Input &     
  Address bus \\ \hline

  PWDATA &      
  \textit{dataWidth} & 
  Input &     
  Write data bus driven when PWRITE is HIGH\\ \hline

  PRDATA &      
  \textit{dataWidth} & 
  Output &     
  Read data bus driven when PWRITE is LOW\\ \hline
 
  PREADY &        
  1 & 
  Output &       
  Transfer ready \\ \hline

  PSLVERR &        
  1 & 
  Output &       
  Transfer error \\ \hline
  
\end{longtable}
\captionsetup{aboveskip=0pt}
\captionof{table}{Apb Ports Descriptions}\label{table:interface}
\endgroup
