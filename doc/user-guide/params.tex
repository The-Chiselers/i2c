\section{Parameter Descriptions}

Below is an example table for top-level parameters controlling the I2C core’s configuration. Adjust as needed.

\renewcommand*{\arraystretch}{1.4}
\begingroup
\small
\rowcolors{2}{gray!30}{gray!10}
\arrayrulecolor{gray!50}

\begin{longtable}[H]{
    | p{0.22\textwidth}
    | p{0.12\textwidth}
    | p{0.05\textwidth}
    | p{0.05\textwidth}
    | p{0.50\textwidth} |
  }
  \hline
  \rowcolor{black}
  \textcolor{white}{\textbf{Name}} &
  \textcolor{white}{\textbf{Type}} &
  \textcolor{white}{\textbf{Min}} &
  \textcolor{white}{\textbf{Max}} &
  \textcolor{white}{\textbf{Description}} \\ 
  \hline \hline
  \endfirsthead

  \textcolor{white}{\textbf{Name}} &
  \textcolor{white}{\textbf{Type}} &
  \textcolor{white}{\textbf{Min}} &
  \textcolor{white}{\textbf{Max}} &
  \textcolor{white}{\textbf{Description}} \\ 
  \hline \hline
  \endhead

  \hline
  \endfoot

clkFreq &
Integer &
1 &
- &
The main system clock frequency in MHz used by the I2C core. \\ \hline

i2cSpeed &
Integer &
100 &
1000 &
I2C bus speed in kHz (e.g., 100 for Standard mode, 400 for Fast mode, 1000 for Fast+).\\ \hline

addrWidth &
Integer &
7 &
10 &
Address width: typically 7-bit addressing, optionally support 10-bit. \\ \hline

\caption{I2C Core Parameter Descriptions}
\label{table:params}
\end{longtable}
\endgroup
