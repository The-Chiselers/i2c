\section{Parameter Descriptions}

The parameters for \textbf{I2C} are shown below in
Table 3.

\renewcommand*{\arraystretch}{1.4}
\begingroup
\small
\rowcolors{2}{gray!30}{gray!10}
\arrayrulecolor{gray!50}

\begin{longtable}[H]{
  | p{0.25\textwidth}
  | p{0.10\textwidth}
  | p{0.05\textwidth}
  | p{0.05\textwidth}
  | p{0.47\textwidth} |
}
\hline
\rowcolor{gray}

\textcolor{white}{\textbf{Name}} &
\textcolor{white}{\textbf{Type}} &
\textcolor{white}{\textbf{Min}} &
\textcolor{white}{\textbf{Max}} &
\textcolor{white}{\textbf{Description}} \\ \hline
\endfirsthead

\textcolor{white}{\textbf{Name}} &
\textcolor{white}{\textbf{Type}} &
\textcolor{white}{\textbf{Min}} &
\textcolor{white}{\textbf{Max}} &
\textcolor{white}{\textbf{Description}}            \\ \hline
\endhead


\endfoot

clkFreq &
Integer &
1 &
- &
The main system clock frequency in MHz used by the I2C core. \\ \hline

dataWidth &
Integer &
8 &
32 &
Transmissions can be 8, 16, or 32 bits in size. Used by The PWDATA, PRDATA, mdata, and sdata. \\ \hline

addrWidth     &
Int           &
1             &
$\leq$ 32       &
The Apb address bus width  \\ \hline

regWidth &
Integer &
8 &
8 &
Size of control and interrupt registers. Maximum number of addressable slaves is $2^8 - 1$.

\end{longtable}
\captionsetup{aboveskip=0pt}
\captionof{table}{Parameter Descriptions}\label{table:params}
\endgroup

The I2C is instantiated into a design as follows:

\begin{lstlisting}[language=Scala]

  // Valid I2C Instantiation Example
  val myI2C = new I2C(
    clkFreq   = 50,
    dataWidth = 32, 
    addrWidth =  8, 
    regWidth  =  8) 

  \end{lstlisting}

