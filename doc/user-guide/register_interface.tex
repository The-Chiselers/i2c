\section{Register Interface}

The I2C module uses an APB interface for register access. Each register starts on a byte address, with the size indicated in the table below. For registers smaller than the data width, the upper bits are unused during writes and read as zero.

\renewcommand*{\arraystretch}{1.4}
\begingroup
\small
\rowcolors{2}{gray!30}{gray!10} % Alternating colors start from the second row
\arrayrulecolor{gray!50}
\begin{longtable}[H]{
  | p{0.27\textwidth}
  | p{0.18\textwidth}
  | p{0.50\textwidth} |
  }
  \hline
  \rowcolor{gray}

  \textcolor{white}{\textbf{Name}} &   
  \textcolor{white}{\textbf{Size (Bits)}} &   
  \textcolor{white}{\textbf{Description}} \\ \hline \hline
  \endfirsthead

  \textcolor{white}{\textbf{Name}} &   
  \textcolor{white}{\textbf{Size (Bits)}} &   
  \textcolor{white}{\textbf{Description}} \\ \hline \hline
  \endhead

  
  MCTRL  &   
  8 &   
  Master Control Register. Contains bits to enable I2C master operation, control ACK/NACK generation, and perform clock stretching. \\ \hline

  MSTATUS &   
  8 &   
  Master Status Register. Contains status flags for operation completion, bus state, arbitration status, and slave ACK response. \\ \hline

  MBAUD &   
  8 &   
  Master Baud Rate Register. Controls the SCL clock frequency when operating as a master. \\ \hline

  MADDR &   
  8 &   
  Master Address Register. Holds the slave address and R/W bit for the next transaction. Writing to this register triggers a START condition. \\ \hline

  MDATA &   
  dataWidth &   
  Master Data Register. Provides access to the master shift register for transmitting or receiving data. \\ \hline

  SCTRL  &   
  8 &   
  Slave Control Register. Contains bits to enable I2C slave operation, control ACK/NACK generation, and perform clock stretching. \\ \hline

  SSTATUS &   
  8 &   
  Slave Status Register. Contains status flags for data transfer completion, address recognition, direction, and slave clock hold status. \\ \hline

  SADDR &   
  8 &   
  Slave Address Register. Holds the 7-bit address that the slave responds to. \\ \hline

  SDATA &   
  dataWidth &   
  Slave Data Register. Provides access to the slave shift register for transmitting or receiving data. \\ \hline

\end{longtable}
\captionsetup{aboveskip=0pt}
\captionof{table}{Register Interface}\label{table:register}

\subsection{Register Operation}

\subsubsection{Master Control (\texttt{MCTRL})}
\label{sec:mctrl}
  
\begin{table}[H]
    \centering
    \caption{Master Control (\texttt{MCTRL})}
    \begin{tabular}{@{}cccccccccc@{}}
        \toprule
        \textbf{Bit} & 7 & 6 & 5 & 4 & 3 & 2 & 1 & 0 \\ \midrule
        \textbf{Name} & N/A & N/A & N/A & SCLH & N/A & N/A & ACKACT & ENABLE \\ \bottomrule
    \end{tabular}
    \label{tab:mctrl}
\end{table}

\begin{itemize}
    \item \textbf{Bit 4 - SCLH (SCL Hold):} 
    \begin{itemize}
        \item \texttt{0}: Normal SCL operation
        \item \texttt{1}: Hold SCL line low to pause communication
    \end{itemize}
    \textit{Description:} When set, the master stretches the SCL clock by holding it low. This can be used to temporarily pause communication when the master needs more time to process data before continuing. The CLKHOLD status bit will be set in MSTATUS when this bit is active.
    
    \item \textbf{Bit 1 - ACKACT (Acknowledge Action):} 
    \begin{itemize}
        \item \texttt{0}: Send ACK in response to received data
        \item \texttt{1}: Send NACK in response to received data
    \end{itemize}
    \textit{Description:} Controls whether the master sends ACK or NACK when receiving data from a slave. Typically set to NACK for the last byte to indicate end of transfer.
    
    \item \textbf{Bit 0 - ENABLE (Enable Master):} 
    \begin{itemize}
        \item \texttt{0}: Master functionality disabled
        \item \texttt{1}: Master functionality enabled
    \end{itemize}
    \textit{Description:} Enables the I2C master functionality. This bit must be set before initiating any master operations.
\end{itemize}

\subsubsection{Master Status (\texttt{MSTATUS})}
\label{sec:mstatus}

\begin{table}[H]
    \centering
    \caption{Master Status (\texttt{MSTATUS})}
    \begin{tabular}{@{}cccccccc@{}}
        \toprule
        \textbf{Bit} & 7 & 6 & 5 & 4 & 3 & 2 & 1-0 \\ \midrule
        \textbf{Name} & RIF & WIF & CLKHOLD & RXACK & N/A & ARBLOST & BUSSTATE[1:0] \\ \bottomrule
    \end{tabular}
    \label{tab:mstatus}
\end{table}

\begin{itemize}
    \item \textbf{Bit 7 - RIF (Read Interrupt Flag):} 
    \begin{itemize}
        \item \texttt{0}: No read operation completed
        \item \texttt{1}: Master read operation completed
    \end{itemize}
    \textit{Description:} Set when a master read operation is complete. Cleared by writing a '1' to this bit, writing to MADDR, or reading/writing MDATA.
    
    \item \textbf{Bit 6 - WIF (Write Interrupt Flag):} 
    \begin{itemize}
        \item \texttt{0}: No write operation completed
        \item \texttt{1}: Master write operation completed
    \end{itemize}
    \textit{Description:} Set when a master address or data write operation is complete. Cleared by writing a '1' to this bit, writing to MADDR, or reading/writing MDATA.
    
    \item \textbf{Bit 5 - CLKHOLD (Clock Hold):} 
    \begin{itemize}
        \item \texttt{0}: SCL operating normally
        \item \texttt{1}: SCL being held low by master
    \end{itemize}
    \textit{Description:} Indicates that the master is currently stretching the clock by holding SCL low.
    
    \item \textbf{Bit 4 - RXACK (Received Acknowledge):} 
    \begin{itemize}
        \item \texttt{0}: ACK received
        \item \texttt{1}: NACK received
    \end{itemize}
    \textit{Description:} Indicates whether an ACK or NACK was received from the slave in response to an address or data transmission.
    
    \item \textbf{Bit 2 - ARBLOST (Arbitration Lost):} 
    \begin{itemize}
        \item \texttt{0}: No arbitration lost
        \item \texttt{1}: Arbitration lost to another master
    \end{itemize}
    \textit{Description:} Indicates that this master lost arbitration to another master on the bus. Cleared by writing a '1' to this bit.
    
    \item \textbf{Bits 1-0 - BUSSTATE[1:0] (Bus State):} 
    \begin{itemize}
        \item \texttt{00}: Reserved
        \item \texttt{01}: IDLE - Bus is free
        \item \texttt{10}: OWNER - This master owns the bus
        \item \texttt{11}: BUSY - Another master owns the bus
    \end{itemize}
    \textit{Description:} Indicates the current state of the I2C bus from this master's perspective.
\end{itemize}

\subsubsection{Master Baud Rate (\texttt{MBAUD})}
\label{sec:mbaud}

\begin{table}[H]
    \centering
    \caption{Master Baud Rate (\texttt{MBAUD})}
    \begin{tabular}{@{}cc@{}}
        \toprule
        \textbf{Bit} & 7 - 0 \\ \midrule
        \textbf{Name} & BAUD \\ \bottomrule
    \end{tabular}
    \label{tab:mbaud}
\end{table}

\begin{itemize}
    \item \textbf{Bits 7-0 - BAUD[7:0] (Baud Rate):} 
    \textit{Description:} This register controls the I2C clock frequency when operating as a master. The clock frequency is determined by the formula: fscl = fclk / (10 + 2 × BAUD), where fclk is the system clock frequency in MHz. It is recommended to set this register while the master is disabled.
\end{itemize}

\subsubsection{Master Address (\texttt{MADDR})}
\label{sec:maddr}

\begin{table}[H]
  \centering
  \caption{Master Address (\texttt{MADDR})}
  \begin{tabular}{@{}cc@{}}
      \toprule
      \textbf{Bit} & 7 - 0 \\ \midrule
      \textbf{Name} & ADDR \\ \bottomrule
  \end{tabular}
  \label{tab:maddr}
\end{table}

\begin{itemize}
  
  \item \textbf{Bits 7-0 - ADDR[7:0] (Address):} 
  \textit{Description:} Contains the 7-bit slave address (bits 7:1) and the R/W direction bit (bit 0). Writing to this register triggers a START condition if the master is enabled and the bus is available. The master will then transmit this address on the bus and wait for an ACK from the addressed slave. If bit 0 is 0, a write operation follows; if bit 0 is 1, a read operation follows.
\end{itemize}

\subsubsection{Master Data (\texttt{MDATA})}
\label{sec:mdata}
  
\begin{table}[H]
  \centering
  \caption{Master Data (\texttt{MDATA})}
  \begin{tabular}{@{}cc@{}}
      \toprule
      \textbf{Bit} & dataWidth-1 - 0 \\ \midrule
      \textbf{Name} & DATA \\ \bottomrule
  \end{tabular}
  \label{tab:mdata}
\end{table}

\begin{itemize}
  
  \item \textbf{Bits dataWidth-1:0 - DATA:} 
  \textit{Description:} This register provides access to the internal shift register used for data transmission and reception. In write mode, writing to this register loads the data to be transmitted. In read mode, reading this register returns the last received data. Reading or writing this register clears the interrupt flags in MSTATUS.
\end{itemize}

%Slave Registers
\subsubsection{Slave Control (\texttt{SCTRL})}
\label{sec:sctrl}

\begin{table}[H]
    \centering
    \caption{Slave Control (\texttt{SCTRL})}
    \begin{tabular}{@{}cccccccccc@{}}
        \toprule
        \textbf{Bit} & 7 & 6 & 5 & 4 & 3 & 2 & 1 & 0 \\ \midrule
        \textbf{Name} & N/A & N/A & N/A & SCLH & N/A & N/A & ACKACT & ENABLE \\ \bottomrule
    \end{tabular}
    \label{tab:sctrl}
\end{table}

\begin{itemize}
    \item \textbf{Bit 4 - SCLH (SCL Hold):} 
    \begin{itemize}
        \item \texttt{0}: Normal SCL operation
        \item \texttt{1}: Hold SCL line low
    \end{itemize}
    \textit{Description:} When set, the slave stretches the SCL clock by holding it low. This can be used when the slave needs more processing time before continuing with data transfer. The CLKHOLD status bit will be set in SSTATUS when this bit is active.
    
    \item \textbf{Bit 1 - ACKACT (Acknowledge Action):} 
    \begin{itemize}
        \item \texttt{0}: Send ACK in response to received data
        \item \texttt{1}: Send NACK in response to received data
    \end{itemize}
    \textit{Description:} Controls whether the slave sends ACK or NACK when receiving data from a master. Typically set to NACK when the slave cannot accept more data.
    
    \item \textbf{Bit 0 - ENABLE (Enable Slave):} 
    \begin{itemize}
        \item \texttt{0}: Slave functionality disabled
        \item \texttt{1}: Slave functionality enabled
    \end{itemize}
    \textit{Description:} Enables the I2C slave functionality. When enabled, the slave will monitor the bus for its address.
\end{itemize}

\subsubsection{Slave Status (\texttt{SSTATUS})}
\label{sec:sstatus}

\begin{table}[H]
    \centering
    \caption{Slave Status (\texttt{SSTATUS})}
    \begin{tabular}{@{}ccccccccc@{}}
        \toprule
        \textbf{Bit} & 7 & 6 & 5 & 4 & 3 & 2 & 1 & 0 \\ \midrule
        \textbf{Name} & DIF & APIF & CLKHOLD & RXACK & N/A & N/A & DIR & AP \\ \bottomrule
    \end{tabular}
    \label{tab:sstatus}
\end{table}

\begin{itemize}
    \item \textbf{Bit 7 - DIF (Data Interrupt Flag):} 
    \begin{itemize}
        \item \texttt{0}: No data transfer completed
        \item \texttt{1}: Data transfer completed
    \end{itemize}
    \textit{Description:} Set when a slave data transfer (transmit or receive) is complete. Cleared by writing a '1' to this bit or reading/writing SDATA.
    
    \item \textbf{Bit 6 - APIF (Address or Stop Interrupt Flag):} 
    \begin{itemize}
        \item \texttt{0}: No address match or STOP condition
        \item \texttt{1}: Address match or STOP condition detected
    \end{itemize}
    \textit{Description:} Set when an address match or STOP condition is detected. The AP bit indicates which event occurred. Cleared by writing a '1' to this bit.
    
    \item \textbf{Bit 5 - CLKHOLD (Clock Hold):} 
    \begin{itemize}
        \item \texttt{0}: SCL operating normally
        \item \texttt{1}: SCL being held low by slave
    \end{itemize}
    \textit{Description:} Indicates that the slave is currently stretching the clock by holding SCL low.
    
    \item \textbf{Bit 4 - RXACK (Received Acknowledge):} 
    \begin{itemize}
        \item \texttt{0}: ACK received
        \item \texttt{1}: NACK received
    \end{itemize}
    \textit{Description:} Indicates whether an ACK or NACK was received from the master in response to data transmitted by the slave.
    
    \item \textbf{Bit 1 - DIR (Read/Write Direction):} 
    \begin{itemize}
        \item \texttt{0}: Master write operation (slave receives)
        \item \texttt{1}: Master read operation (slave transmits)
    \end{itemize}
    \textit{Description:} Indicates the direction of data transfer as determined by the R/W bit in the address packet from the master.
    
    \item \textbf{Bit 0 - AP (Address or Stop):} 
    \begin{itemize}
        \item \texttt{0}: STOP condition detected
        \item \texttt{1}: Address match detected
    \end{itemize}
    \textit{Description:} When APIF is set, this bit indicates whether the interrupt was caused by an address match (1) or a STOP condition (0).
\end{itemize}

\subsubsection{Slave Address (\texttt{SADDR})}
\label{sec:saddr}

\begin{table}[H]
  \centering
  \caption{Slave Address (\texttt{SADDR})}
  \begin{tabular}{@{}cc@{}}
      \toprule
      \textbf{Bit} & 7 - 0 \\ \midrule
      \textbf{Name} & ADDR \\ \bottomrule
  \end{tabular}
  \label{tab:saddr}
\end{table}

\begin{itemize}
  \item \textbf{Bits 7-0 - ADDR[7:0] (Address):} 
  \textit{Description:} Contains the 7-bit slave address that this device will respond to. When an address packet is received from a master, the address is compared with this register. If there is a match, the slave responds with an ACK and sets the AP and APIF bits in SSTATUS.
\end{itemize}

\subsubsection{Slave Data (\texttt{SDATA})}
\label{sec:sdata}

\begin{table}[H]
  \centering
  \caption{Slave Data (\texttt{SDATA})}
  \begin{tabular}{@{}cc@{}}
      \toprule
      \textbf{Bit} & dataWidth-1 - 0 \\ \midrule
      \textbf{Name} & DATA \\ \bottomrule
  \end{tabular}
  \label{tab:sdata}
\end{table}

\begin{itemize}
  \item \textbf{Bits dataWidth-1:0 - DATA:} 
  \textit{Description:} This register provides access to the internal shift register used for slave data transmission and reception. In slave transmit mode, writing to this register loads the data to be transmitted to the master. In slave receive mode, reading this register returns the data received from the master. Reading or writing this register clears the interrupt flags in SSTATUS.
\end{itemize}

\subsection{Register Addresses}

\paragraph{dataWidth: 8}
\begin{table}[H]
  \centering
  \begin{tabular}{|c|c|c|}
      \hline
      \rowcolor{darkgray}  % Dark grey background for header row
      \textcolor{white}{\textbf{Register Name}} & \textcolor{white}{\textbf{Address Start}} & \textcolor{white}{\textbf{Address End}} \\ \hline
      MCTRL & 0x0 & 0x0 \\ \hline
      MSTATUS & 0x1 & 0x1 \\ \hline
      MBAUD & 0x2 & 0x2 \\ \hline
      MADDR & 0x3 & 0x3 \\ \hline
      MDATA & 0x4 & 0x4 \\ \hline
      SCTRL & 0x5 & 0x5 \\ \hline
      SSTATUS & 0x6 & 0x6 \\ \hline
      SADDR & 0x7 & 0x7 \\ \hline
      SDATA & 0x8 & 0x8 \\ \hline
  \end{tabular}
  \caption{8-bit Register Addressing}
\end{table}

\paragraph{dataWidth: 16}
\begin{table}[H]
  \centering
  \begin{tabular}{|c|c|c|}
    \hline
    \rowcolor{darkgray}  % Dark grey background for header row
    \textcolor{white}{\textbf{Register Name}} & \textcolor{white}{\textbf{Address Start}} & \textcolor{white}{\textbf{Address End}} \\ \hline
    MCTRL & 0x0 & 0x0 \\ \hline
    MSTATUS & 0x1 & 0x1 \\ \hline
    MBAUD & 0x2 & 0x2 \\ \hline
    MADDR & 0x3 & 0x3 \\ \hline
    MDATA & 0x4 & 0x5 \\ \hline
    SCTRL & 0x6 & 0x6 \\ \hline
    SSTATUS & 0x7 & 0x7 \\ \hline
    SADDR & 0x8 & 0x8 \\ \hline
    SDATA & 0x9 & 0xA \\ \hline
  \end{tabular}
  \caption{16-bit Register Addressing}
\end{table}

\paragraph{dataWidth: 32}
\begin{table}[H]
  \centering
  \begin{tabular}{|c|c|c|}
    \hline
    \rowcolor{darkgray}  % Dark grey background for header row
    \textcolor{white}{\textbf{Register Name}} & \textcolor{white}{\textbf{Address Start}} & \textcolor{white}{\textbf{Address End}} \\ \hline
    MCTRL & 0x0 & 0x0 \\ \hline
    MSTATUS & 0x1 & 0x1 \\ \hline
    MBAUD & 0x2 & 0x2 \\ \hline
    MADDR & 0x3 & 0x3 \\ \hline
    MDATA & 0x4 & 0x7 \\ \hline
    SCTRL & 0x8 & 0x8 \\ \hline
    SSTATUS & 0x9 & 0x9 \\ \hline
    SADDR & 0xA & 0xA \\ \hline
    SDATA & 0xB & 0xE \\ \hline
  \end{tabular}
  \caption{32-bit Register Addressing}
\end{table}