% chktex-file 44
\subsection{Register Interface}
 
When programming registers, each register starts on a byte address, and the last bits it would take up in its final byte based on its size are unused. To find the size in bytes for any register, divide by the register size, and round up to the nearest whole number. For example, a 32-bit register would take up 4 bytes, and a 1-bit register would take up 1 byte.
\renewcommand*{\arraystretch}{1.4}
\begingroup
\small
\rowcolors{2}{gray!30}{gray!10} % Alternating colors start from the second row
\arrayrulecolor{gray!50}
\begin{longtable}[H]{
  | p{0.27\textwidth}
  | p{0.18\textwidth}
  | p{0.50\textwidth} |
  }
  \hline
  \rowcolor{gray}

  \textcolor{white}{\textbf{Name}} &   
  \textcolor{white}{\textbf{Size (Bits)}} &   
  \textcolor{white}{\textbf{Description}} \\ \hline \hline
  \endfirsthead

  \textcolor{white}{\textbf{Name}} &   
  \textcolor{white}{\textbf{Size (Bits)}} &   
  \textcolor{white}{\textbf{Description}} \\ \hline \hline
  \endhead

  
  MCTRL  &   
  8 &   
  DESC TODO \\ \hline

  MSTATUS &   
  8 &   
  DESC TODO \\ \hline

  MBAUD &   
  8 &   
  DESC TODO \\ \hline

  MADDR &   
  8 &   
  DESC TODO \\ \hline

  MDATA &   
  dataWidth &   
  DESC TODO \\ \hline

  SCTRL  &   
  8 &   
  DESC TODO \\ \hline

  SSTATUS &   
  8 &   
  DESC TODO \\ \hline

  SADDR &   
  8 &   
  DESC TODO \\ \hline

  SDATA &   
  dataWidth &   
  DESC TODO \\ \hline

\end{longtable}
\captionsetup{aboveskip=0pt}
\captionof{table}{Register Interface}\label{table:register}

  \newpage

  \subsection{Register Operation}

  \subsubsection{Master Control (\texttt{MCTRL})}
  \label{sec:mctrl}
  
  \begin{table}[H]
      \centering
      \caption{Master Control (\texttt{MCTRL})}
      \begin{tabular}{@{}cccccccccc@{}}
          \toprule
          \textbf{Bit} & 7 & 6 & 5 & 4 & 3 & 2 & 1 & 0 \\ \midrule
          \textbf{Name} & N/A & N/A & N/A & SCLH & N/A & N/A & ACKACT & ENABLE \\ \bottomrule
      \end{tabular}
      \label{tab:mctrl}
  \end{table}
  
  \begin{itemize}
      \item \textbf{Bit 4 - SCLH (SCL Hold):} 
      \begin{itemize}
          \item \texttt{0}: Do NOT hold the SCL line low
          \item \texttt{1}: Hold the SCL line low
      \end{itemize}
      \textit{Description:} This bit allows the user to hold the SCL low indefinetly. This can only be used for pausing operation to write an ACK/NACK to the I2C Slave.
      
      \item \textbf{Bit 1 - ACKAT (Acknowledge Action):} 
      \begin{itemize}
          \item \texttt{0}: Send ACK
          \item \texttt{1}: Send NACK
      \end{itemize}
      \textit{Description:} User can send an ACK/NACK to data recieved from Slave in Master Read Mode.
      
      \item \textbf{Bit 0 - ENABLE (Enable Master):} 
      \begin{itemize}
          \item \texttt{0}: Master is disabled.
          \item \texttt{1}: Master is enabled.
      \end{itemize}
      \textit{Description:} Writing a '1' to this bit will enable the I2C as a master.
  \end{itemize}
  
  \subsubsection{Master Status (\texttt{MSTATUS})}
  \label{sec:mstatus}
  
  \begin{table}[H]
      \centering
      \caption{Master Status (\texttt{MSTATUS})}
      \begin{tabular}{@{}cccccccc@{}}
          \toprule
          \textbf{Bit} & 7 & 6 & 5 & 4 & 3 & 2 & 1-0 \\ \midrule
          \textbf{Name} & RIF & WIF & CLKHOLD & RXACK & N/A & ARBLOST & BUSSTATE[1:0] \\ \bottomrule
      \end{tabular}
      \label{tab:mstatus}
  \end{table}
  
  
  \begin{itemize}
      \item \textbf{Bit 7 - RIF (Read Interrupt Flag):} 
      \begin{itemize}
          \item \texttt{0}: This flag can be cleared by: 1. Writing a '1' to it. 2. Writing to the MADDR register. 3. Writing/Reading the MDATA register.
          \item \texttt{1}: Enabled.
      \end{itemize}
      \textit{Description:} This flag is set to '1' when the master byte read operation is successfully completed and can be cleared using the methods listed above.
      
      \item \textbf{Bit 6 - WIF (Write Interrupt Flag):} 
      \begin{itemize}
          \item \texttt{0}: This flag can be cleared by: 1. Writing a '1' to it. 2. Writing to the MADDR register. 3. Writing/Reading the MDATA register.
          \item \texttt{1}: Enabled.
      \end{itemize}
      \textit{Description:} This flag is set to '1' when the master transmit address or byte write operation is completed, regardless of the occrurence of arbitration lost.
      
      \item \textbf{Bit 5 - CLKHOLD (Clock Hold):} 
      \begin{itemize}
          \item \texttt{0}: Normal clock operation is occuring.
          \item \texttt{1}: Master is currently holding SCL low.
      \end{itemize}
      \textit{Description:} The flag is set to '1' when the master is currently holding the SCL low, stretching the I2C period. This bit can be cleared using the methods from the RIF/WIF flag.

      \item \textbf{Bit 4 - RXACK (Recieved Acknowledge):} 
      \begin{itemize}
          \item \texttt{0}: The most recent Acknowledge bit from the slave was ACK - slave is ready for more data.
          \item \texttt{1}: The most recent Acknowledge bit from the slave was NACK - slave is not able to or does not need to recieve more data.
      \end{itemize}
      \textit{Description:} Status Bit of the Slave Acknowledge state.

      \item \textbf{Bit 2 - ARBLOST (Arbitration Lost):} 
      \begin{itemize}
          \item \texttt{0}: Master still has control of bus.
          \item \texttt{1}: This master has lost arbitration to another master.
      \end{itemize}
      \textit{Description:} Arbitration status of current master. ARBLOST can occur: 1. While transmitting an address packet 2. While writing data to slave. This bit can be cleared using the methods from the RIF/WIF flag.

      \item \textbf{Bits 1-0 - BUSSTATE[1:0] (Bus State):} 
      \begin{itemize}
          \item \texttt{00}: UNKOWN.
          \item \texttt{01}: IDLE - Master is currently in the Idle state.
          \item \texttt{10}: OWNER - Master is currently in control of the bus.
          \item \texttt{11}: BUSY - Another Master currently has control of the bus.
      \end{itemize}
      \textit{Description:} Status of the I2C bus, useful for multi-master settings and arbitration handling.
  \end{itemize}
  
  \subsubsection{Master Baud Rate (\texttt{MBAUD})}
  \label{sec:mbaud}
  
  \begin{table}[H]
      \centering
      \caption{Master Baud Rate (\texttt{MBAUD})}
      \begin{tabular}{@{}cc@{}}
          \toprule
          \textbf{Bit} & 7 - 0 \\ \midrule
          \textbf{Name} & BAUD \\ \bottomrule
      \end{tabular}
      \label{tab:mbaud}
  \end{table}
  
  \begin{itemize}
      
      \item \textbf{Bits 7-0 - BAUD[7:0] (Baud Rate):} 
      \textit{Description:} This bit field is used to derive the SCL high and low time. It must be written while the master is disabled. The master
      can be disabled by writing ‘0’ to the Enable TWI Master (ENABLE) bit from the Master Control Register. fscl = fclk / (10 + 2 x BAUD)
  \end{itemize}

  \subsubsection{Master Address (\texttt{MADDR})}
  \label{sec:maddr}

  \begin{table}[H]
    \centering
    \caption{Master Address (\texttt{MADDR})}
    \begin{tabular}{@{}cc@{}}
        \toprule
        \textbf{Bit} & 7 - 0 \\ \midrule
        \textbf{Name} & ADDR \\ \bottomrule
    \end{tabular}
    \label{tab:maddr}
  \end{table}

  \begin{itemize}
    
    \item \textbf{Bits 7-0 - ADDR[7:0] (Address):} 
    \textit{Description:} This register contains the address of the external slave device. When this bit field is written, the I2C will issue a Start
    condition, and the shift register performs a byte transmit operation on the bus depending on the bus state.
    This register can be read at any time without interfering with the ongoing bus activity since a read access does not
    trigger the master logic to perform any bus protocol related operations.
    The master control logic uses the bit 0 of this register as the R/W direction bit
  \end{itemize}

    \subsubsection{Master Data (\texttt{MDATA})}
    \label{sec:mdata}
  
    \begin{table}[H]
      \centering
      \caption{Master Data (\texttt{MDATA})}
      \begin{tabular}{@{}cc@{}}
          \toprule
          \textbf{Bit} & 7 - 0 \\ \midrule
          \textbf{Name} & DATA \\ \bottomrule
      \end{tabular}
      \label{tab:mdata}
    \end{table}
  
    \begin{itemize}
      
      \item \textbf{Bits 7-0 - DATA[7:0] (Data):} 
      \textit{Description:} This bit field provides direct access to the master’s physical shift register, which is used to shift out data on the bus
      (transmit) and to shift in data received from the bus (receive). The direct access implies that the MDATA register
      cannot be accessed during byte transmissions.
    \end{itemize}

    %Slave Registers
    \subsubsection{Slave Control (\texttt{SCTRL})}
    \label{sec:sctrl}
    
    \begin{table}[H]
        \centering
        \caption{Slave Control (\texttt{SCTRL})}
        \begin{tabular}{@{}cccccccccc@{}}
            \toprule
            \textbf{Bit} & 7 & 6 & 5 & 4 & 3 & 2 & 1 & 0 \\ \midrule
            \textbf{Name} & N/A & N/A & N/A & SCLH & N/A & N/A & ACKACT & ENABLE \\ \bottomrule
        \end{tabular}
        \label{tab:mctrl}
    \end{table}
    
    \begin{itemize}
        \item \textbf{Bit 4 - SCLH (SCL Hold):} 
        \begin{itemize}
            \item \texttt{0}: Do NOT hold the SCL line low
            \item \texttt{1}: Hold the SCL line low
        \end{itemize}
        \textit{Description:} This bit allows the user to hold the SCL low indefinetly. This can only be used for pausing operation to write an ACK/NACK to the I2C Master.
        
        \item \textbf{Bit 1 - ACKAT (Acknowledge Action):} 
        \begin{itemize}
            \item \texttt{0}: Send ACK
            \item \texttt{1}: Send NACK
        \end{itemize}
        \textit{Description:} User can send an ACK/NACK to data recieved from Master in Slave Read Mode.
        
        \item \textbf{Bit 0 - ENABLE (Enable Slave):} 
        \begin{itemize}
            \item \texttt{0}: Slave is disabled.
            \item \texttt{1}: Slave is enabled.
        \end{itemize}
        \textit{Description:} Writing a '1' to this bit will enable the I2C as a Slave.
    \end{itemize}
    
    \subsubsection{Slave Status (\texttt{SSTATUS})}
    \label{sec:sstatus}
    
    \begin{table}[H]
        \centering
        \caption{Slave Status (\texttt{SSTATUS})}
        \begin{tabular}{@{}ccccccccc@{}}
            \toprule
            \textbf{Bit} & 7 & 6 & 5 & 4 & 3 & 2 & 1 & 0 \\ \midrule
            \textbf{Name} & DIF & APIF & CLKHOLD & RXACK & N/A & N/A & DIR & AP \\ \bottomrule
        \end{tabular}
        \label{tab:sstatus}
    \end{table}
    
    
    \begin{itemize}
        \item \textbf{Bit 7 - DIF (Data Interrupt Flag):} 
        \begin{itemize}
            \item \texttt{0}: This flag can be cleared by: 1. Writing a '1' to it. 2. Writing/Reading the SDATA register.
            \item \texttt{1}: Enabled.
        \end{itemize}
        \textit{Description:} This flag is set to '1' when the slave byte transmit or recieve operation is completed.
        
        \item \textbf{Bit 6 - APIF (Address or Stop Interrupt Flag):} 
        \begin{itemize}
            \item \texttt{0}: This flag can be cleared using the same methods in DIF.
            \item \texttt{1}: Enabled.
        \end{itemize}
        \textit{Description:} This flag is set to '1' when the slave address has been recieved, or by a Stop condition. 
        
        \item \textbf{Bit 5 - CLKHOLD (Clock Hold):} 
        \begin{itemize}
            \item \texttt{0}: Normal clock operation is occuring.
            \item \texttt{1}: Slave is currently holding SCL low.
        \end{itemize}
        \textit{Description:} The flag is set to '1' when the slave is currently holding the SCL low, stretching the I2C period. This bit can be cleared using the methods from the DIF flag.
  
        \item \textbf{Bit 4 - RXACK (Recieved Acknowledge):} 
        \begin{itemize}
            \item \texttt{0}: The most recent Acknowledge bit from the master was ACK - master is ready for more data.
            \item \texttt{1}: The most recent Acknowledge bit from the master was NACK - master is not able to or does not need to recieve more data.
        \end{itemize}
        \textit{Description:} Status Bit of the Slave Acknowledge state.
  
        \item \textbf{Bit 1 - DIR (Read/Write Direction):} 
        \begin{itemize}
            \item \texttt{0}: Master write operation is in progress.
            \item \texttt{1}: Master read operation is in progress.
        \end{itemize}
        \textit{Description:} This bit reflects the direction bit value from the last address packet recieved from the master device.
  
        \item \textbf{Bit 0 - AP (Address or Stop):} 
        \begin{itemize}
          \item \texttt{0}: A stop condition generated the interrupt on the APIF flag.
          \item \texttt{1}: Address detection generated the interrupt on the APIF flag.
        \end{itemize}
        \textit{Description:} This bit determines whether the APID interrupt is due to an address detection or a Stop condition.
    \end{itemize}
    
    \subsubsection{Slave Address (\texttt{SADDR})}
    \label{sec:saddr}
  
    \begin{table}[H]
      \centering
      \caption{Slave Address (\texttt{SADDR})}
      \begin{tabular}{@{}cc@{}}
          \toprule
          \textbf{Bit} & 7 - 0 \\ \midrule
          \textbf{Name} & ADDR \\ \bottomrule
      \end{tabular}
      \label{tab:saddr}
    \end{table}
  
    \begin{itemize}
      
      \item \textbf{Bits 7-0 - ADDR[7:0] (Address):} 
      \textit{Description:} The Slave Address register is used by the slave address match logic to determine if a master device
      has addressed the slave. The Address or Stop Interrupt Flag (APIF) and the Address or Stop (AP) bit in the
      Slave Status register are set to ‘1’ if an address packet is received.
    \end{itemize}
  
      \subsubsection{Slave Data (\texttt{SDATA})}
      \label{sec:sdata}
    
      \begin{table}[H]
        \centering
        \caption{Slave Data (\texttt{SDATA})}
        \begin{tabular}{@{}cc@{}}
            \toprule
            \textbf{Bit} & 7 - 0 \\ \midrule
            \textbf{Name} & DATA \\ \bottomrule
        \end{tabular}
        \label{tab:sdata}
      \end{table}
    
      \begin{itemize}
        
        \item \textbf{Bits 7-0 - DATA[7:0] (Data):} 
        \textit{Description:} This bit field provides access to the slave data register.
      \end{itemize}
  
  \subsection{Register Addresses}
  
  \paragraph{dataWidth: 8}
  \begin{table}[H]
    \centering
    \begin{tabular}{|c|c|c|}
        \hline
        \rowcolor{darkgray}  % Dark grey background for header row
        \textcolor{white}{\textbf{Register Name}} & \textcolor{white}{\textbf{Address Start}} & \textcolor{white}{\textbf{Address End}} \\ \hline
        MCTRL & 0x0 & 0x0 \\ \hline
        MSTATUS & 0x1 & 0x1 \\ \hline
        MBAUD & 0x2 & 0x2 \\ \hline
        MADDR & 0x3 & 0x3 \\ \hline
        MDATA & 0x4 & 0x4 \\ \hline
        SCTRL & 0x5 & 0x5 \\ \hline
        SSTATUS & 0x6 & 0x6 \\ \hline
        SADDR & 0x7 & 0x7 \\ \hline
        SDATA & 0x8 & 0x8 \\ \hline
    \end{tabular}
    \caption{8-bit Register Addressing}
  \end{table}
  
  \paragraph{dataWidth: 16}
  \begin{table}[H]
    \centering
    \begin{tabular}{|c|c|c|}
      \hline
      \rowcolor{darkgray}  % Dark grey background for header row
      \textcolor{white}{\textbf{Register Name}} & \textcolor{white}{\textbf{Address Start}} & \textcolor{white}{\textbf{Address End}} \\ \hline
      MCTRL & 0x0 & 0x0 \\ \hline
      MSTATUS & 0x1 & 0x1 \\ \hline
      MBAUD & 0x2 & 0x2 \\ \hline
      MADDR & 0x3 & 0x3 \\ \hline
      MDATA & 0x4 & 0x5 \\ \hline
      SCTRL & 0x6 & 0x6 \\ \hline
      SSTATUS & 0x7 & 0x7 \\ \hline
      SADDR & 0x8 & 0x8 \\ \hline
      SDATA & 0x9 & 0xA \\ \hline
    \end{tabular}
    \caption{16-bit Register Addressing}
  \end{table}
  
  \paragraph{dataWidth: 32}
  \begin{table}[H]
    \centering
    \begin{tabular}{|c|c|c|}
      \hline
      \rowcolor{darkgray}  % Dark grey background for header row
      \textcolor{white}{\textbf{Register Name}} & \textcolor{white}{\textbf{Address Start}} & \textcolor{white}{\textbf{Address End}} \\ \hline
      MCTRL & 0x0 & 0x0 \\ \hline
      MSTATUS & 0x1 & 0x1 \\ \hline
      MBAUD & 0x2 & 0x2 \\ \hline
      MADDR & 0x3 & 0x3 \\ \hline
      MDATA & 0x4 & 0x6 \\ \hline
      SCTRL & 0x7 & 0x7 \\ \hline
      SSTATUS & 0x8 & 0x8 \\ \hline
      SADDR & 0x9 & 0x9 \\ \hline
      SDATA & 0xA & 0xC \\ \hline
    \end{tabular}
    \caption{32-bit Register Addressing}
  \end{table}