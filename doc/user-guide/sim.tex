\section{Simulation}

\subsection{Tests}
The test bench generates a number (default is 2) configurations of the
I2C that are highly randomized. The user can run the full test-suite  
with n-configurations using "sbt "test" -DtestName="regression"".

User can also run individual tests or test groups by substituting "regression" with
the relevant test name. A description of the tests is below:

\begin{table}[h]
  \resizebox{\textwidth}{!}{
  \centering
  \begin{tabular}{|c|c|c|c|}
      \hline
      \rowcolor{gray}  % Dark grey background for header row
      \textcolor{white}{\textbf{Test Group Name}} & \textcolor{white}{\textbf{Test Name}} & \textcolor{white}{\textbf{Test Type}} & \textcolor{white}{\textbf{Test Description}} \\ \hline
      clockTests & masterClock & Directed & Generate the correct clock frequency for master mode \\ \hline
      clockTests & dividerBasic & Directed & Verify the Divider-based I2C clock generation in a simple scenario \\ \hline
      clockTests & dividerRandom & Directed & Verify random BAUD values do produce toggles \\ \hline 
      arbitrationTests & arbitrationLost & Directed & Handle multi-master arbitration lost \\ \hline  
      abitrationTests & multiMasterExtended & Random & Handle multi-master arbitration and extra transmission check \\ \hline
      repeatedStartTests & repeatedStart & Directed & Perform repeated start with 16 or 32-bit data bus widths \\ \hline
      transmitTests & masterSlaveTransmission & Directed & Transmit data between master and slave \\ \hline
      transmitTests & masterWriteSlaveReadFullDuplex & Directed & Perform Master Write->Slave Read in FullDuplexI2C \\ \hline
      transmitTests & bidirectionalHalfDuplex & Random & Perform bidirectional half-duplex communication, check interrupt flags \\ \hline
      transmitTests & bidirectionalHalfDuplexTwice & Random & Perform bidirectional half-duplex communication across different baud rates \\ \hline
      transmitTests & ackVsNackFullDuplex & Directed & Check scenario for ack vs. NACK in FullDuplexI2C \\ \hline
      transmitTests & stopConditionFullDuplex & Directed & Verify Stop condition logic from Master->Slave in FullDuplexI2C \\ \hline
      transmitTests & noSlavePresentFullDuplex & Directed & Check Master sees NACK if no slave present in FullDuplexI2C\\ \hline
      transmitTests & clockStretchingSlave & Random & Check if Master sees NACK and ACK during Clock Stretching \\ \hline
      transmitTests & clockStretchingMaster & Random & Check if Slave sees NACK and ACK during Clock Stretching \\ \hline
    \end{tabular}
  }
  \caption{Test Suite}
\end{table}

\subsection{Toggle Coverage}
Current Score: 92.41\%

All inputs and outputs are checked to insure each toggle at least once. If coverage is enabled
during core instantiation, a cumulative coverage report of all tests is generated under ./out/cov/verliog.

Exceptions are higher bits of the \emph{PADDR}, \emph{PWDATA}, and \emph{PRDATA}
which are intended to be static during each simulation. These signals are
excluded from coverage checks.

\subsection{Code Coverage}
Current Score: 83.58\%

Code coverage for all tests can be generated as follows:
\begin{verbatim}
  $ sbt coverage test -DtestName="allTests"
  $ sbt coverageReport
  $ python3 -m http.server 8000 --directory target/scoverage-report/
  View report on local host: http://localhost:8000/index.html
\end{verbatim}

\subsection{Running simulation}

Simulations can be run directly from the command prompt as follows:

\begin{verbatim}
  $ sbt test -DtestName="allTests"
\end{verbatim}

or from make as follows:

\texttt{\$ make test}

\subsection{Viewing the waveforms}

The simulation generates an FST file that can be viewed using a waveform viewer. The command to view the waveform is as follows:
\begin{verbatim}
  $ gtkwave ./out/test/i2c.fst
\end{verbatim}
