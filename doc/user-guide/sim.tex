\section{Simulation}

\subsection{Tests}
The test bench for the I2C module generates multiple configurations to verify functionality under various conditions:

\begin{itemize}
    \item Master-only tests that verify address and data transmission
    \item Slave response tests that check address recognition and data handling
    \item Multi-master tests that verify correct arbitration
    \item Clock stretching tests to ensure proper SCL handling
    \item Error condition tests that verify proper error detection and handling
\end{itemize}

The tests include both directed tests with predetermined transactions and randomized tests to explore edge cases and verify robustness.

\subsection{Code Coverage}
The test environment measures coverage across all I2C module functions:

\begin{itemize}
    \item All state machine transitions
    \item All register fields
    \item All control signals and flags
    \item All error conditions and recovery mechanisms
\end{itemize}

The coverage metrics ensure that the design is thoroughly verified before synthesis. All inputs and outputs are checked to ensure toggling at least once, with an error thrown if any port fails to toggle.

\subsection{Running Simulation}
Simulations can be run directly from the command prompt:

\begin{verbatim}
$ sbt "test:runMain tech.rocksavage.chiselware.I2C.I2CTest"
\end{verbatim}

Or using make:

\begin{verbatim}
$ make test
\end{verbatim}

\subsection{Viewing Waveforms}
Generated waveforms can be viewed using standard waveform viewers:

\begin{verbatim}
$ gtkwave ./out/test/i2c_waves.vcd
\end{verbatim}

The waveforms show all interface signals including master/slave bus operations, internal state transitions, and register access activity.