\section{Usage Examples}

\subsection{Master Write Example}
The following example demonstrates a typical master write operation:

\begin{verbatim}
// 1. Configure master
write(MBAUD, 0x0A);         // Configure for 100kHz
write(MCTRL, 0x01);         // Enable master

// 2. Send data to slave device at address 0x50
write(MADDR, 0xA0);         // Address 0x50 (7-bit) with Write bit (0)
                            // This will issue START condition

// 3. Wait for address transmission to complete
while ((read(MSTATUS) & 0x40) == 0); // Wait for WIF flag

// 4. Check if slave acknowledged
if ((read(MSTATUS) & 0x10) != 0) {
    // NACK received, handle error
    // ...
    return;
}

// 5. Send data bytes
write(MDATA, 0x12);         // Send first data byte
while ((read(MSTATUS) & 0x40) == 0); // Wait for WIF flag

write(MDATA, 0x34);         // Send second data byte
while ((read(MSTATUS) & 0x40) == 0); // Wait for WIF flag

// 6. Done (STOP condition automatically sent)
\end{verbatim}

\subsection{Master Read Example}
The following example demonstrates a typical master read operation:

\begin{verbatim}
// 1. Configure master
write(MBAUD, 0x0A);         // Configure for 100kHz
write(MCTRL, 0x01);         // Enable master

// 2. Request data from slave device at address 0x50
write(MADDR, 0xA1);         // Address 0x50 (7-bit) with Read bit (1)
                            // This will issue START condition

// 3. Wait for address transmission to complete
while ((read(MSTATUS) & 0x40) == 0); // Wait for WIF flag

// 4. Check if slave acknowledged
if ((read(MSTATUS) & 0x10) != 0) {
    // NACK received, handle error
    // ...
    return;
}

// 5. Read first data byte (ACK)
while ((read(MSTATUS) & 0x80) == 0); // Wait for RIF flag
uint8_t data1 = read(MDATA);

// 6. Read second data byte (NACK to end transmission)
write(MCTRL, 0x03);         // Set ACKACT to NACK
while ((read(MSTATUS) & 0x80) == 0); // Wait for RIF flag
uint8_t data2 = read(MDATA);

// 7. Done (STOP condition automatically sent)
\end{verbatim}

\subsection{Slave Operation Example}
The following example demonstrates slave mode operation:

\begin{verbatim}
// 1. Configure slave
write(SADDR, 0x50);         // Set slave address
write(SCTRL, 0x01);         // Enable slave

// 2. Wait for address match
while(true) {
    uint8_t status = read(SSTATUS);
    
    // Check if address match has occurred
    if ((status & 0x41) == 0x41) {  // APIF and AP bits set
        // Address match detected
        
        // 3. Determine direction (read/write)
        if ((status & 0x02) == 0) {
            // Master Write, Slave Read
            handleSlaveReceive();
        } else {
            // Master Read, Slave Write
            handleSlaveTransmit();
        }
    }
    
    // 4. Check if transaction has ended (STOP condition)
    if ((status & 0x41) == 0x40) {  // APIF set but AP clear
        // Transaction ended with STOP
        break;
    }
}

// Slave Receive handler
void handleSlaveReceive() {
    while(true) {
        // Wait for data
        while ((read(SSTATUS) & 0x80) == 0); // Wait for DIF flag
        
        // Read data
        uint8_t data = read(SDATA);
        
        // Process data...
        
        // Check if transaction ended
        if ((read(SSTATUS) & 0x41) == 0x40)
            break;
    }
}

// Slave Transmit handler
void handleSlaveTransmit() {
    // Prepare first data byte
    write(SDATA, 0x55);
    
    while(true) {
        // Wait for byte to be transmitted
        while ((read(SSTATUS) & 0x80) == 0); // Wait for DIF flag
        
        // Check if master has NACKed (end of transaction)
        if ((read(SSTATUS) & 0x10) != 0)
            break;
            
        // Prepare next data byte
        write(SDATA, 0xAA);
        
        // Check if transaction ended
        if ((read(SSTATUS) & 0x41) == 0x40)
            break;
    }
}
\end{verbatim}