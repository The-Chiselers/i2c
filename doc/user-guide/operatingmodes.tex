\section{Operating Modes}

The I2C core supports multiple roles and modes.

\subsection{Master Mode}
In Master mode, the I2C core generates the SCL clock, issues Start/Stop conditions, and orchestrates bus communication.

\begin{itemize}
    \item \textbf{Single or Multi-Master:} Multi-master requires hardware arbitration to handle collisions.
    \item \textbf{Clock Generation:} A baud-rate generator sets the SCL frequency within Standard/Fast/Fast+ ranges.
    \item \textbf{Address Transmission:} The master sets the slave’s 7-bit address and the R/W direction bit, then waits for an ACK.
\end{itemize}

\subsection{Slave Mode}
In Slave mode, the I2C core monitors the bus for its own address, then responds accordingly:

\begin{itemize}
    \item \textbf{Address Match:} If the incoming address matches the core's SADDR register, the slave acknowledges.
    \item \textbf{Read/Write Handling:} The \texttt{DIR} bit (or equivalent) indicates whether the master is reading or writing to the slave.
    \item \textbf{Clock Stretching:} The slave can hold SCL low to delay the master if needed.
\end{itemize}

